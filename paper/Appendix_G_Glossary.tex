\section*{Appendix G: Glossary}
\addcontentsline{toc}{section}{Appendix G: Glossary}

\begin{description}

\item[Abstention:]  
The deliberate withholding of a response when structural validation or resonance cannot be achieved, as opposed to providing potentially misleading or unverifiable content.

\item[Adversarial Prompt / Ambiguous Prompt:]  
A query or instruction designed to test the boundaries of the model by presenting conflicting, ambiguous, or misleading information. Used in evaluation to probe for hallucination or validation failures.

\item[Axiomatic Validation:]  
Systematic evaluation of each operation or output against the explicitly defined axioms of the COMPASS framework. Requires logical justification based on formal principles rather than heuristic or probabilistic patterns.

\item[Baseline Model / Standard LLM:]  
A language model evaluated without COMPASS-specific prompts or structural validation logic. Used as a point of comparison to assess the structural effects of the COMPASS framework.

\item[Empirical Validation:]  
Evaluation through controlled experiments, benchmarks, or real-world tests, in contrast to purely theoretical or simulated validation.

\item[Factuality:]  
The degree to which an output can be verified against validated knowledge domains or resonance fields. In COMPASS, factuality is a prerequisite for any output to be produced as a factual answer.

\item[Goal Principles:]  
High-level objectives that guide dynamic adaptation and overall system alignment. Distinguished from axioms by their broader, often value-oriented, system-shaping character (e.g., ZP-008: systemic coherence).

\item[Human Evaluation:]  
The assessment of model outputs by independent human evaluators, typically measuring factuality, coherence, and transparency.

\item[Hypothesis Marking / Explicit Hypothesis:]  
The explicit labeling of an output as hypothetical when the system cannot validate it against axioms or resonance fields. Used to distinguish unverifiable content from factual statements.

\item[Justified Refusal / Justification:]  
A model response in which content is withheld and an explicit, principled reason is given, typically referencing a violated axiom, resonance failure, or lack of structural grounding.

\item[Meta-Reflection:]  
A higher-order process by which the system, upon encountering ambiguity or contradiction, initiates internal self-evaluation, potentially deriving temporary sub-axioms or adjusting its evaluation logic in accordance with overarching goal principles.

\item[Overarching Goal Principles:]  
A subset of goal principles that provide system-wide priorities or meta-objectives, potentially governing the resolution of conflicts between individual axioms or other principles.

\item[Probabilistic (Model, Interpretation, Exclusion):]  
Describes the inherently statistical nature of large language models, where outputs are generated according to learned token distributions rather than deterministic rules. In the context of COMPASS, it denotes the limitation that exclusion of hallucination is achieved only as a behavioral effect, not as a mathematical guarantee.

\item[Prompt-Based Implementation / Prompt-Based Instruction:]  
A method where system behavior is guided by specific textual instructions (prompts) provided to the underlying language model. In COMPASS, this represents the current operational mode, with all limitations and interpretive risks of LLM prompting.

\item[Reflection / Reflexive Validation:]  
Continuous monitoring and assessment of the system’s own operations, ensuring outputs remain consistent with axioms and systemic coherence. See also A7 (Reflection).

\item[Resonance Field:]  
Any domain of pre-validated, coherent information against which generated outputs are matched for consistency and meaningful connection.

\item[Structure-Driven Mechanism:]  
Any operational process within the model that is governed by explicit axioms, goal principles, or validation logic, rather than by statistical or probabilistic rules alone.

\item[Validation Pipeline / Validation Path:]  
The structured sequence of checks and logical steps that each potential output undergoes, from initial evaluation against axioms to resonance testing and projection validation.

\item[Withholding Output:]  
The system’s refusal to produce an output when validation fails. Distinguished from hypothesis marking by the absence of even a hypothetical answer.

\end{description}
