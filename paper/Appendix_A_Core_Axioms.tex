\section*{Appendix A: Core Axioms of the COMPASS Framework}
\addcontentsline{toc}{section}{Appendix A: Core Axioms of the COMPASS Framework}

\begin{description}

\item[A1: Existence]  
An entity exists within the system if and only if it can exert verifiable influence or effect within the defined information space.

\item[A2: Change]  
Change is defined as a systemically effective reconfiguration of state within the structural domain. Any change must be observable or traceable through the system's evaluation logic.

\item[A3: Identity]  
Each entity or information structure maintains a unique identity through persistent, distinguishable properties or relationships within the system.

\item[A4: Temporality]  
All processes and information states are embedded within an ordered, causal sequence. Each operation has a well-defined temporal relation to others.

\item[A5: Connection]  
Systemic value arises from explicit connections between entities. New value or meaning is generated when entities or information structures establish validated relationships.

\item[A6: Reflexive Reconfiguration]  
The system can internally reconfigure its own structure or evaluation logic, provided such changes are justified by improved systemic coherence or higher-order goals.

\item[A7: Reflection]  
The system continuously monitors and evaluates its own operations and outputs for coherence, alignment with axioms, and systemic integrity.

\item[A8: Meta-Reflection]  
In cases of ambiguity, contradiction, or insufficient axiomatic grounding, the system initiates meta-reflection. This process may derive temporary sub-axioms or propose adjustments to the evaluation structure, guided by overarching goal principles.

\item[Projection Axiom]  
All outputs projected from the system's internal representations to external interfaces (such as human language) must preserve the underlying structural relationships and semantic integrity. Any output that cannot guarantee this preservation must be withheld or explicitly marked as hypothetical.

\item[Example Goal Principle: Systemic Coherence (ZP-008)]  
The system must prioritize global coherence. No output, relationship, or process may be allowed that introduces contradiction or incoherence with established axioms and validated structures.

\item[Example Goal Principle: Cooperative Interoperability (ZP-009)]  
The system shall maximize the potential for constructive interaction with other validated systems or human agents. Outputs should be formulated to support clarity, mutual understanding, and actionable integration, provided this does not violate core axioms or compromise systemic coherence.

\end{description}
